%% Generated by Sphinx.
\def\sphinxdocclass{report}
\documentclass[letterpaper,10pt,english,openany,oneside]{sphinxmanual}
\ifdefined\pdfpxdimen
   \let\sphinxpxdimen\pdfpxdimen\else\newdimen\sphinxpxdimen
\fi \sphinxpxdimen=.75bp\relax

\PassOptionsToPackage{warn}{textcomp}
\usepackage[utf8]{inputenc}
\ifdefined\DeclareUnicodeCharacter
% support both utf8 and utf8x syntaxes
  \ifdefined\DeclareUnicodeCharacterAsOptional
    \def\sphinxDUC#1{\DeclareUnicodeCharacter{"#1}}
  \else
    \let\sphinxDUC\DeclareUnicodeCharacter
  \fi
  \sphinxDUC{00A0}{\nobreakspace}
  \sphinxDUC{2500}{\sphinxunichar{2500}}
  \sphinxDUC{2502}{\sphinxunichar{2502}}
  \sphinxDUC{2514}{\sphinxunichar{2514}}
  \sphinxDUC{251C}{\sphinxunichar{251C}}
  \sphinxDUC{2572}{\textbackslash}
\fi
\usepackage{cmap}
\usepackage[T1]{fontenc}
\usepackage{amsmath,amssymb,amstext}
\usepackage{babel}



\usepackage{times}
\expandafter\ifx\csname T@LGR\endcsname\relax
\else
% LGR was declared as font encoding
  \substitutefont{LGR}{\rmdefault}{cmr}
  \substitutefont{LGR}{\sfdefault}{cmss}
  \substitutefont{LGR}{\ttdefault}{cmtt}
\fi
\expandafter\ifx\csname T@X2\endcsname\relax
  \expandafter\ifx\csname T@T2A\endcsname\relax
  \else
  % T2A was declared as font encoding
    \substitutefont{T2A}{\rmdefault}{cmr}
    \substitutefont{T2A}{\sfdefault}{cmss}
    \substitutefont{T2A}{\ttdefault}{cmtt}
  \fi
\else
% X2 was declared as font encoding
  \substitutefont{X2}{\rmdefault}{cmr}
  \substitutefont{X2}{\sfdefault}{cmss}
  \substitutefont{X2}{\ttdefault}{cmtt}
\fi


\usepackage[Bjarne]{fncychap}
\usepackage{sphinx}

\fvset{fontsize=\small}
\usepackage{geometry}


% Include hyperref last.
\usepackage{hyperref}
% Fix anchor placement for figures with captions.
\usepackage{hypcap}% it must be loaded after hyperref.
% Set up styles of URL: it should be placed after hyperref.
\urlstyle{same}
\addto\captionsenglish{\renewcommand{\contentsname}{Contents:}}

\usepackage{sphinxmessages}
\setcounter{tocdepth}{1}



\title{Banker's Algorithm Documentation}
\date{Apr 07, 2020}
\release{}
\author{Chezka Gaddi}
\newcommand{\sphinxlogo}{\vbox{}}
\renewcommand{\releasename}{}
\makeindex
\begin{document}

\pagestyle{empty}
\sphinxmaketitle
\pagestyle{plain}
\sphinxtableofcontents
\pagestyle{normal}
\phantomsection\label{\detokenize{index::doc}}



\chapter{Introduction}
\label{\detokenize{intro:introduction}}\label{\detokenize{intro::doc}}

\section{Overview}
\label{\detokenize{intro:overview}}
This is a multi\sphinxhyphen{}threaded program that implements the Banker’s Algorithm. Several customers request and release resources from the bank. The banker will grant a request only if it leaves the system in a safe state. A request that leaves the system in an unsafe state will be denied. This programming assignment combines three separate topics:
(1) multi\sphinxhyphen{}threading, (2) preventing race conditions, and (3) deadlock avoidance.


\section{Running the Application}
\label{\detokenize{intro:running-the-application}}
The program is built by running \sphinxstyleemphasis{make} in the main project folder.

The program is invoked by passing the number of resources of each type on the command line.

\begin{sphinxVerbatim}[commandchars=\\\{\}]
./bankers \PYG{l+m}{10} \PYG{l+m}{5} \PYG{l+m}{7}
\end{sphinxVerbatim}

The \sphinxstyleemphasis{available} array is initialized with these values. Currently, the program will take in initial values for three resources.


\chapter{Implementation}
\label{\detokenize{implementation:implementation}}\label{\detokenize{implementation::doc}}

\section{Banker’s Algorithm}
\label{\detokenize{implementation:banker-s-algorithm}}

\chapter{Implementation}
\label{\detokenize{source_code:implementation}}\label{\detokenize{source_code::doc}}

\section{bankers.c}
\label{\detokenize{source_code:bankers-c}}

\begin{fulllineitems}
\pysigline{\sphinxbfcode{\sphinxupquote{void~init(const~char~*{[}{]}~argv)}}}
Initialize all relevant data structures and synchronization objects.

Initializes the available array to the command line inputs, the
allocation array to 0, the maximum array to a random number dependent on the
available array and the need array to the maximum \sphinxhyphen{} allocation.

The mutex lock is also initialized as well as the customer ID numbers.
\begin{quote}\begin{description}
\item[{Parameters}] \leavevmode\begin{itemize}
\item {} 
\sphinxstyleliteralstrong{\sphinxupquote{argv}} \textendash{} command line inputs of the number of available resources.

\end{itemize}

\end{description}\end{quote}

\end{fulllineitems}

\index{create\_customers (C function)@\spxentry{create\_customers}\spxextra{C function}}

\begin{fulllineitems}
\phantomsection\label{\detokenize{source_code:c.create_customers}}%
\pysigstartmultiline
\pysiglinewithargsret{void \sphinxbfcode{\sphinxupquote{create\_customers}}}{}{}%
\pysigstopmultiline
Create all of the customer threads.

\end{fulllineitems}



\begin{fulllineitems}
\pysigline{\sphinxbfcode{\sphinxupquote{int~safety\_test(int~customer,~int~{[}{]}~request)}}}
safety\_test first makes a copy of the current state to run tests on. It then
applies the request by subtracting the request array from the allocation
array and adds the request array to the customer’s allocation array. It then
follows the Banker’s Algorithm with the safety test.
\begin{quote}\begin{description}
\item[{Parameters}] \leavevmode\begin{itemize}
\item {} 
\sphinxstyleliteralstrong{\sphinxupquote{customer}} \textendash{} number of the customer making the request

\item {} 
\sphinxstyleliteralstrong{\sphinxupquote{request}} \textendash{} resources being requested

\end{itemize}

\item[{Returns}] \leavevmode
0 if safe state is found, \sphinxhyphen{}1 if safe state not found

\end{description}\end{quote}

\end{fulllineitems}

\index{print\_test\_state (C function)@\spxentry{print\_test\_state}\spxextra{C function}}

\begin{fulllineitems}
\phantomsection\label{\detokenize{source_code:c.print_test_state}}%
\pysigstartmultiline
\pysiglinewithargsret{void \sphinxbfcode{\sphinxupquote{print\_test\_state}}}{}{}%
\pysigstopmultiline
Prints the contents of the test arrays.

\end{fulllineitems}

\index{print\_state (C function)@\spxentry{print\_state}\spxextra{C function}}

\begin{fulllineitems}
\phantomsection\label{\detokenize{source_code:c.print_state}}%
\pysigstartmultiline
\pysiglinewithargsret{void \sphinxbfcode{\sphinxupquote{print\_state}}}{}{}%
\pysigstopmultiline
Prints the current state of the system.

\end{fulllineitems}



\begin{fulllineitems}
\pysigline{\sphinxbfcode{\sphinxupquote{int~main(int~argc,~const~char~*{[}{]}~argv)}}}
Initializes the matrices, prints the initial state of the system and syncs
the customer threads.

\end{fulllineitems}



\section{customers.c}
\label{\detokenize{source_code:customers-c}}\index{customer\_loop (C function)@\spxentry{customer\_loop}\spxextra{C function}}

\begin{fulllineitems}
\phantomsection\label{\detokenize{source_code:c.customer_loop}}%
\pysigstartmultiline
\pysiglinewithargsret{void *\sphinxbfcode{\sphinxupquote{customer\_loop}}}{void *\sphinxstyleemphasis{param}}{}%
\pysigstopmultiline
The customer loop first creates a request for a random number of resources
dependent on their need. They send the request and keep sending that request
until it is either met, or they reached the max number of request.

The loop also releases a random number resources, and releases all of of the
customer’s allocated resources when their need goes to 0.
\begin{quote}\begin{description}
\item[{Parameters}] \leavevmode\begin{itemize}
\item {} 
\sphinxstyleliteralstrong{\sphinxupquote{param}} \textendash{} The customer id

\end{itemize}

\item[{Returns}] \leavevmode
0 on success

\end{description}\end{quote}

\end{fulllineitems}



\begin{fulllineitems}
\pysigline{\sphinxbfcode{\sphinxupquote{int~request\_resources(int~customer\_num,~int~{[}{]}~request)}}}
request\_resources obtains the mutex lock to stage sending the request for
resources to the bank. If the request is more than what is currently
available or if it exceeds the needs of the customer, the request fails.

It sends the request to the safety\_test to ensure that the request is safe
and sends the request through when approved.

It then prints the current state of the system and unlocks the mutex.
\begin{quote}\begin{description}
\item[{Parameters}] \leavevmode\begin{itemize}
\item {} 
\sphinxstyleliteralstrong{\sphinxupquote{customer\_num}} \textendash{} number of requesting customer

\item {} 
\sphinxstyleliteralstrong{\sphinxupquote{request}} \textendash{} number of resources to be requested

\end{itemize}

\item[{Returns}] \leavevmode
0 if request went through, \sphinxhyphen{}1 if request is denied

\end{description}\end{quote}

\end{fulllineitems}



\begin{fulllineitems}
\pysigline{\sphinxbfcode{\sphinxupquote{int~release\_resources(int~customer\_num,~int~{[}{]}~release)}}}
release\_resources locks the mutex and then adds the release array to the
available array and subtracts the release array to the customer’s allocation
array.
\begin{quote}\begin{description}
\item[{Parameters}] \leavevmode\begin{itemize}
\item {} 
\sphinxstyleliteralstrong{\sphinxupquote{customer\_num}} \textendash{} number of the customer releasing resources

\item {} 
\sphinxstyleliteralstrong{\sphinxupquote{release}} \textendash{} number of resources to be released

\end{itemize}

\item[{Returns}] \leavevmode
0 if successful, \sphinxhyphen{}1 if unsuccessful

\end{description}\end{quote}

\end{fulllineitems}

\index{calculate\_need (C function)@\spxentry{calculate\_need}\spxextra{C function}}

\begin{fulllineitems}
\phantomsection\label{\detokenize{source_code:c.calculate_need}}%
\pysigstartmultiline
\pysiglinewithargsret{int \sphinxbfcode{\sphinxupquote{calculate\_need}}}{int \sphinxstyleemphasis{customer\_num}}{}%
\pysigstopmultiline
Calculates the total number of resources that a given customer still needs.
\begin{quote}\begin{description}
\item[{Parameters}] \leavevmode\begin{itemize}
\item {} 
\sphinxstyleliteralstrong{\sphinxupquote{customer\_num}} \textendash{} customer number to be evaluated

\end{itemize}

\item[{Returns}] \leavevmode
the total number of resources the customer still needs

\end{description}\end{quote}

\end{fulllineitems}



\chapter{Indices and tables}
\label{\detokenize{index:indices-and-tables}}\begin{itemize}
\item {} 
\DUrole{xref,std,std-ref}{genindex}

\item {} 
\DUrole{xref,std,std-ref}{modindex}

\item {} 
\DUrole{xref,std,std-ref}{search}

\end{itemize}



\renewcommand{\indexname}{Index}
\printindex
\end{document}